\nonstopmode
\documentclass[12pt]{report} 
\usepackage{hyperref}
\hypersetup{colorlinks=true,linkcolor=blue}
\usepackage{theorem,graphicx}
\usepackage{listings,alltt}
\bibliographystyle{plain}


\lstset{ %configuring the display of R codes
             tabsize=4,
             language=R,
             basicstyle=\ttfamily,
             aboveskip={1\baselineskip},
             showstringspaces=false,
             breaklines=true,
             showspaces=false,
             numbers=left,
             numberstyle=\small,
             stringstyle=\normalfont,
             keywordstyle=\color{red},
             emph={clc, all, gca},
             emphstyle=\color{red},
             commentstyle=\color{blue}\normalfont}

 
% code environment
{\theorembodyfont{\rmfamily} \newtheorem{codemass}{R code}[chapter]}
\newenvironment{code}%
{\begin{codemass}}{\hrule \end{codemass}}

{\theorembodyfont{\rmfamily} \newtheorem{accmass}{Acc}[chapter]}
\newenvironment{acc-code}%
{\begin{accmass}}{\end{accmass}}


% create listing for code

\newcommand\tcaption[1]
     {\addcontentsline{cod}{section}{\protect\numberline {\thecodemass}#1}}
\makeatletter \newcommand\listofcode
     {\chapter*{List of R Codes\markboth%
                        {\bf List of R Codes}{}}%
\renewcommand*\l@section{\@dottedtocline{1}{1.5em}{5em}}%
\addcontentsline{toc}{chapter}{\protect\numberline{List of R Codes}}
\@starttoc{cod}}
\newcommand\l@matlab[3]
     {#1 \par\noindent#2, #3 \par}
\renewcommand\@pnumwidth{2.1em}
%\makeatother

\makeatletter
\def\curlable#1{\def\thecodemass{#1}\def\@currentlabel{#1}}
\makeatother

\newcommand{\coderef}[1]{Exa~\ref{#1}}
\newcommand{\figref}[1]{Fig.~\ref{#1}}\title{R Textbook Companion for \\Numerical Methods in Finance and Economics: A MATLAB-Based Introduction\\by Paolo Brandimarte\footnote{Funded by a grant from the National Mission on Education through ICT - \href{http://spoken-tutorial.org/NMEICT-Intro}{http://spoken-tutorial.org/NMEICT-Intro}. This Textbook Companion and R codes written in it can be downloaded from the "Textbook Companion Project" section at the website - \href{https://r.fossee.in}{https://r.fossee.in}.}}
\author{ Created by \\Bhushan Sahadeo Manjarekar\\B.E.\\Others\\University of Mumbai\\ Cross-Checked by \\R TBC Team\\}
\date{\today}
\begin{document}
\maketitle

\chapter*{Book Description}
\begin{description}
\item [Title:] Numerical Methods in Finance and Economics: A MATLAB-Based Introduction
\item [Author:] Paolo Brandimarte
\item [Publisher:] John Wiley \& Sons, Inc., Hoboken, New Jersey
\item [Edition:] 2
\item [Year:] 2006
\item [ISBN:] 0471745030
\end{description}

\newpage
\vspace*{3cm}
R numbering policy used in this document and the relation to the above book.
\begin{description}
\item[Exa] Example (Solved example)
\item[Eqn] Equation (Particular equation of the above book)
\end{description}
For example, Exa~3.51 means solved example 3.51 of this book. Sec~2.3 means an R code whose theory is explained in Section 2.3 of the book.

\tableofcontents
\listofcode


\setcounter{chapter}{1}
\chapter{ Financial Theory}

\vspace*{10mm}
\curlable{Exa~2.7}
\begin{code}
\tcaption{Basic theory of interest rates}{Basic theory of interest rates}
\lstinputlisting{../Numerical_Methods_In_Finance_And_Economics:_A_Matlab-Based_Introduction_by_Paolo_Brandimarte/CH2/EX2.7/Ex2_7.R}
\end{code}

\vspace*{10mm}
\curlable{Exa~2.8}
\begin{code}
\tcaption{Basic theory of interest rates}{Basic theory of interest rates}
\lstinputlisting{../Numerical_Methods_In_Finance_And_Economics:_A_Matlab-Based_Introduction_by_Paolo_Brandimarte/CH2/EX2.8/Ex2_8.R}
\end{code}

\vspace*{10mm}
\curlable{Exa~2.9}
\begin{code}
\tcaption{Basic pricing of fixed income securities}{Basic pricing of fixed income securities}
\lstinputlisting{../Numerical_Methods_In_Finance_And_Economics:_A_Matlab-Based_Introduction_by_Paolo_Brandimarte/CH2/EX2.9/Ex2_9.R}
\end{code}

\vspace*{10mm}
\curlable{Exa~2.10}
\begin{code}
\tcaption{Interest rate sensitivity and bond portfolio immunization}{Interest rate sensitivity and bond portfolio immunization}
\lstinputlisting{../Numerical_Methods_In_Finance_And_Economics:_A_Matlab-Based_Introduction_by_Paolo_Brandimarte/CH2/EX2.10/Ex2_10.R}
\end{code}

\vspace*{10mm}
\curlable{Exa~2.24}
\begin{code}
\tcaption{Black Scholes model in MATLAB}{Black Scholes model in MATLAB}
\lstinputlisting{../Numerical_Methods_In_Finance_And_Economics:_A_Matlab-Based_Introduction_by_Paolo_Brandimarte/CH2/EX2.24/Ex2_24.R}
\end{code}

\chapter{Basics of Numerical Analysis}

\vspace*{10mm}
\curlable{Exa~3.2}
\begin{code}
\tcaption{Error propagation conditioning and instability}{Error propagation conditioning and instability}
\lstinputlisting{../Numerical_Methods_In_Finance_And_Economics:_A_Matlab-Based_Introduction_by_Paolo_Brandimarte/CH3/EX3.2/Ex3_2.R}
\end{code}

\vspace*{10mm}
\curlable{Exa~3.4}
\begin{code}
\tcaption{Vector and matrix norms}{Vector and matrix norms}
\lstinputlisting{../Numerical_Methods_In_Finance_And_Economics:_A_Matlab-Based_Introduction_by_Paolo_Brandimarte/CH3/EX3.4/Ex3_4.R}
\end{code}

\vspace*{10mm}
\curlable{Exa~3.6}
\begin{code}
\tcaption{Vector and matrix norms}{Vector and matrix norms}
\lstinputlisting{../Numerical_Methods_In_Finance_And_Economics:_A_Matlab-Based_Introduction_by_Paolo_Brandimarte/CH3/EX3.6/Ex3_6.R}
\end{code}

\vspace*{10mm}
\curlable{Exa~3.7}
\begin{code}
\tcaption{Condition number for a matrix}{Condition number for a matrix}
\lstinputlisting{../Numerical_Methods_In_Finance_And_Economics:_A_Matlab-Based_Introduction_by_Paolo_Brandimarte/CH3/EX3.7/Ex3_7.R}
\end{code}

\vspace*{10mm}
\curlable{Exa~3.8}
\begin{code}
\tcaption{Condition number for a matrix}{Condition number for a matrix}
\lstinputlisting{../Numerical_Methods_In_Finance_And_Economics:_A_Matlab-Based_Introduction_by_Paolo_Brandimarte/CH3/EX3.8/Ex3_8.R}
\end{code}

\vspace*{10mm}
\curlable{Exa~3.11}
\begin{code}
\tcaption{Direct methods for solving systems of linear equations}{Direct methods for solving systems of linear equations}
\lstinputlisting{../Numerical_Methods_In_Finance_And_Economics:_A_Matlab-Based_Introduction_by_Paolo_Brandimarte/CH3/EX3.11/Ex3_11.R}
\end{code}

\vspace*{10mm}
\curlable{Exa~3.12}
\begin{code}
\tcaption{Direct methods for solving systems of linear equations}{Direct methods for solving systems of linear equations}
\lstinputlisting{../Numerical_Methods_In_Finance_And_Economics:_A_Matlab-Based_Introduction_by_Paolo_Brandimarte/CH3/EX3.12/Ex3_12.R}
\end{code}

\vspace*{10mm}
\curlable{Exa~3.13}
\begin{code}
\tcaption{Iterative methods for solving systems of linear equations}{Iterative methods for solving systems of linear equations}
\lstinputlisting{../Numerical_Methods_In_Finance_And_Economics:_A_Matlab-Based_Introduction_by_Paolo_Brandimarte/CH3/EX3.13/Ex3_13.R}
\end{code}

\vspace*{10mm}
\curlable{Exa~3.14}
\begin{code}
\tcaption{Iterative methods for solving systems of linear equations}{Iterative methods for solving systems of linear equations}
\lstinputlisting{../Numerical_Methods_In_Finance_And_Economics:_A_Matlab-Based_Introduction_by_Paolo_Brandimarte/CH3/EX3.14/Ex3_14.R}
\end{code}

\vspace*{10mm}
\curlable{Exa~3.15}
\begin{code}
\tcaption{FUNCTION APPROXIMATION AND INTERPOLATION}{FUNCTION APPROXIMATION AND INTERPOLATION}
\lstinputlisting{../Numerical_Methods_In_Finance_And_Economics:_A_Matlab-Based_Introduction_by_Paolo_Brandimarte/CH3/EX3.15/Ex3_15.R}
\end{code}

\vspace*{10mm}
\curlable{Exa~3.16}
\begin{code}
\tcaption{Elementary polynomial interpolation}{Elementary polynomial interpolation}
\lstinputlisting{../Numerical_Methods_In_Finance_And_Economics:_A_Matlab-Based_Introduction_by_Paolo_Brandimarte/CH3/EX3.16/Ex3_16.R}
\end{code}

\vspace*{10mm}
\curlable{Exa~3.17}
\begin{code}
\tcaption{Elementary polynomial interpolation}{Elementary polynomial interpolation}
\lstinputlisting{../Numerical_Methods_In_Finance_And_Economics:_A_Matlab-Based_Introduction_by_Paolo_Brandimarte/CH3/EX3.17/Ex3_17.R}
\end{code}

\vspace*{10mm}
\curlable{Exa~3.18}
\begin{code}
\tcaption{Interpolation by cubic splines}{Interpolation by cubic splines}
\lstinputlisting{../Numerical_Methods_In_Finance_And_Economics:_A_Matlab-Based_Introduction_by_Paolo_Brandimarte/CH3/EX3.18/Ex3_18.R}
\end{code}

\vspace*{10mm}
\curlable{Exa~3.22}
\begin{code}
\tcaption{Bisection method}{Bisection method}
\lstinputlisting{../Numerical_Methods_In_Finance_And_Economics:_A_Matlab-Based_Introduction_by_Paolo_Brandimarte/CH3/EX3.22/Ex3_22.R}
\end{code}

\vspace*{10mm}
\curlable{Exa~3.23}
\begin{code}
\tcaption{Newtons method}{Newtons method}
\lstinputlisting{../Numerical_Methods_In_Finance_And_Economics:_A_Matlab-Based_Introduction_by_Paolo_Brandimarte/CH3/EX3.23/Ex3_23.R}
\end{code}

\vspace*{10mm}
\curlable{Exa~3.24}
\begin{code}
\tcaption{Optimization based solution of non linear equations}{Optimization based solution of non linear equations}
\lstinputlisting{../Numerical_Methods_In_Finance_And_Economics:_A_Matlab-Based_Introduction_by_Paolo_Brandimarte/CH3/EX3.24/Ex3_24.R}
\end{code}

\chapter{ Numerical Integration Deterministic and Monte Carlo Methods}

\vspace*{10mm}
\curlable{Exa~4.1}
\begin{code}
\tcaption{Numerical integration in MATLAB}{Numerical integration in MATLAB}
\lstinputlisting{../Numerical_Methods_In_Finance_And_Economics:_A_Matlab-Based_Introduction_by_Paolo_Brandimarte/CH4/EX4.1/Ex4_1.R}
\end{code}

\vspace*{10mm}
\curlable{Exa~4.2}
\begin{code}
\tcaption{MONTE CARLO INTEGRATION}{MONTE CARLO INTEGRATION}
\lstinputlisting{../Numerical_Methods_In_Finance_And_Economics:_A_Matlab-Based_Introduction_by_Paolo_Brandimarte/CH4/EX4.2/Ex4_2.R}
\end{code}

\vspace*{10mm}
\curlable{Exa~4.3}
\begin{code}
\tcaption{MONTE CARLO INTEGRATION}{MONTE CARLO INTEGRATION}
\lstinputlisting{../Numerical_Methods_In_Finance_And_Economics:_A_Matlab-Based_Introduction_by_Paolo_Brandimarte/CH4/EX4.3/Ex4_3.R}
\end{code}

\vspace*{10mm}
\curlable{Exa~4.4}
\begin{code}
\tcaption{Generating pseudorandom numbers}{Generating pseudorandom numbers}
\lstinputlisting{../Numerical_Methods_In_Finance_And_Economics:_A_Matlab-Based_Introduction_by_Paolo_Brandimarte/CH4/EX4.4/Ex4_4.R}
\end{code}

\vspace*{10mm}
\curlable{Exa~4.5}
\begin{code}
\tcaption{Generating pseudorandom numbers}{Generating pseudorandom numbers}
\lstinputlisting{../Numerical_Methods_In_Finance_And_Economics:_A_Matlab-Based_Introduction_by_Paolo_Brandimarte/CH4/EX4.5/Ex4_5.R}
\end{code}

\vspace*{10mm}
\curlable{Exa~4.6}
\begin{code}
\tcaption{Inverse transform method}{Inverse transform method}
\lstinputlisting{../Numerical_Methods_In_Finance_And_Economics:_A_Matlab-Based_Introduction_by_Paolo_Brandimarte/CH4/EX4.6/Ex4_6.R}
\end{code}

\vspace*{10mm}
\curlable{Exa~4.8}
\begin{code}
\tcaption{Generating normal variates by the polar approach}{Generating normal variates by the polar approach}
\lstinputlisting{../Numerical_Methods_In_Finance_And_Economics:_A_Matlab-Based_Introduction_by_Paolo_Brandimarte/CH4/EX4.8/Ex4_8.R}
\end{code}

\vspace*{10mm}
\curlable{Exa~4.9}
\begin{code}
\tcaption{Generating normal variates by the polar approach}{Generating normal variates by the polar approach}
\lstinputlisting{../Numerical_Methods_In_Finance_And_Economics:_A_Matlab-Based_Introduction_by_Paolo_Brandimarte/CH4/EX4.9/Ex4_9.R}
\end{code}

\vspace*{10mm}
\curlable{Exa~4.10}
\begin{code}
\tcaption{SETTING THE NUMBER OF REPLICATIONS}{SETTING THE NUMBER OF REPLICATIONS}
\lstinputlisting{../Numerical_Methods_In_Finance_And_Economics:_A_Matlab-Based_Introduction_by_Paolo_Brandimarte/CH4/EX4.10/Ex4_10.R}
\end{code}

\vspace*{10mm}
\curlable{Exa~4.11}
\begin{code}
\tcaption{Antithetic sampling}{Antithetic sampling}
\lstinputlisting{../Numerical_Methods_In_Finance_And_Economics:_A_Matlab-Based_Introduction_by_Paolo_Brandimarte/CH4/EX4.11/Ex4_11.R}
\end{code}

\vspace*{10mm}
\curlable{Exa~4.12}
\begin{code}
\tcaption{BlsMCAV}{BlsMCAV}
\lstinputlisting{../Numerical_Methods_In_Finance_And_Economics:_A_Matlab-Based_Introduction_by_Paolo_Brandimarte/CH4/EX4.12/Ex4_12.R}
\end{code}

\vspace*{10mm}
\curlable{Exa~4.14}
\begin{code}
\tcaption{Importance sampling}{Importance sampling}
\lstinputlisting{../Numerical_Methods_In_Finance_And_Economics:_A_Matlab-Based_Introduction_by_Paolo_Brandimarte/CH4/EX4.14/Ex4_14.R}
\end{code}

\vspace*{10mm}
\curlable{Exa~4.15}
\begin{code}
\tcaption{Generating Halton low discrepancy sequences}{Generating Halton low discrepancy sequences}
\lstinputlisting{../Numerical_Methods_In_Finance_And_Economics:_A_Matlab-Based_Introduction_by_Paolo_Brandimarte/CH4/EX4.15/Ex4_15.R}
\end{code}

\vspace*{10mm}
\curlable{Exa~4.16}
\begin{code}
\tcaption{Generating Halton low discrepancy sequences}{Generating Halton low discrepancy sequences}
\lstinputlisting{../Numerical_Methods_In_Finance_And_Economics:_A_Matlab-Based_Introduction_by_Paolo_Brandimarte/CH4/EX4.16/Ex4_16.R}
\end{code}

\vspace*{10mm}
\curlable{Exa~4.17}
\begin{code}
\tcaption{Generating Halton low discrepancy sequences}{Generating Halton low discrepancy sequences}
\lstinputlisting{../Numerical_Methods_In_Finance_And_Economics:_A_Matlab-Based_Introduction_by_Paolo_Brandimarte/CH4/EX4.17/Ex4_17.R}
\end{code}

\vspace*{10mm}
\curlable{Exa~4.18}
\begin{code}
\tcaption{Generating Sobol low discrepancy sequences}{Generating Sobol low discrepancy sequences}
\lstinputlisting{../Numerical_Methods_In_Finance_And_Economics:_A_Matlab-Based_Introduction_by_Paolo_Brandimarte/CH4/EX4.18/Ex4_18.R}
\end{code}

\vspace*{10mm}
\curlable{Exa~4.19}
\begin{code}
\tcaption{Generating Sobol low discrepancy sequences}{Generating Sobol low discrepancy sequences}
\lstinputlisting{../Numerical_Methods_In_Finance_And_Economics:_A_Matlab-Based_Introduction_by_Paolo_Brandimarte/CH4/EX4.19/Ex4_19.R}
\end{code}

\vspace*{10mm}
\curlable{Exa~4.20}
\begin{code}
\tcaption{Generating Sobol low discrepancy sequences}{Generating Sobol low discrepancy sequences}
\lstinputlisting{../Numerical_Methods_In_Finance_And_Economics:_A_Matlab-Based_Introduction_by_Paolo_Brandimarte/CH4/EX4.20/Ex4_20.R}
\end{code}

\chapter{ Finite Difference Methods for Partial Differential Equations}

\vspace*{10mm}
\curlable{Exa~5.1}
\begin{code}
\tcaption{Instability in a finite difference scheme}{Instability in a finite difference scheme}
\lstinputlisting{../Numerical_Methods_In_Finance_And_Economics:_A_Matlab-Based_Introduction_by_Paolo_Brandimarte/CH5/EX5.1/Ex5_1.R}
\end{code}

\vspace*{10mm}
\curlable{Exa~5.3}
\begin{code}
\tcaption{Solving the heat equation by an explicit method}{Solving the heat equation by an explicit method}
\lstinputlisting{../Numerical_Methods_In_Finance_And_Economics:_A_Matlab-Based_Introduction_by_Paolo_Brandimarte/CH5/EX5.3/Ex5_3.R}
\end{code}

\vspace*{10mm}
\curlable{Exa~5.4}
\begin{code}
\tcaption{Solving the heat equation by a fully implicit method}{Solving the heat equation by a fully implicit method}
\lstinputlisting{../Numerical_Methods_In_Finance_And_Economics:_A_Matlab-Based_Introduction_by_Paolo_Brandimarte/CH5/EX5.4/Ex5_4.R}
\end{code}

\chapter{ Convex Optimization}

\vspace*{10mm}
\curlable{Exa~6.1}
\begin{code}
\tcaption{Finite vs infinite dimensional problems}{Finite vs infinite dimensional problems}
\lstinputlisting{../Numerical_Methods_In_Finance_And_Economics:_A_Matlab-Based_Introduction_by_Paolo_Brandimarte/CH6/EX6.1/Ex6_1.R}
\end{code}

\vspace*{10mm}
\curlable{Exa~6.3}
\begin{code}
\tcaption{Linear vs non linear problems}{Linear vs non linear problems}
\lstinputlisting{../Numerical_Methods_In_Finance_And_Economics:_A_Matlab-Based_Introduction_by_Paolo_Brandimarte/CH6/EX6.3/Ex6_3.R}
\end{code}

\vspace*{10mm}
\curlable{Exa~6.5}
\begin{code}
\tcaption{Penalty function approach}{Penalty function approach}
\lstinputlisting{../Numerical_Methods_In_Finance_And_Economics:_A_Matlab-Based_Introduction_by_Paolo_Brandimarte/CH6/EX6.5/Ex6_5.R}
\end{code}

\vspace*{10mm}
\curlable{Exa~6.8}
\begin{code}
\tcaption{Kuhn Tucker conditions}{Kuhn Tucker conditions}
\lstinputlisting{../Numerical_Methods_In_Finance_And_Economics:_A_Matlab-Based_Introduction_by_Paolo_Brandimarte/CH6/EX6.8/Ex6_8.R}
\end{code}

\vspace*{10mm}
\curlable{Exa~6.12}
\begin{code}
\tcaption{Geometric and algebraic features of linear programming}{Geometric and algebraic features of linear programming}
\lstinputlisting{../Numerical_Methods_In_Finance_And_Economics:_A_Matlab-Based_Introduction_by_Paolo_Brandimarte/CH6/EX6.12/Ex6_12.R}
\end{code}

\chapter{ Option Pricing by Binomial and Trinomial Lattices}

\vspace*{10mm}
\curlable{Exa~7.1}
\begin{code}
\tcaption{Calibrating a binomial lattice}{Calibrating a binomial lattice}
\lstinputlisting{../Numerical_Methods_In_Finance_And_Economics:_A_Matlab-Based_Introduction_by_Paolo_Brandimarte/CH7/EX7.1/Ex7_1.R}
\end{code}

\vspace*{10mm}
\curlable{Exa~7.2}
\begin{code}
\tcaption{Accuracy of the binomial lattice for decreasing deltaT}{Accuracy of the binomial lattice for decreasing deltaT}
\lstinputlisting{../Numerical_Methods_In_Finance_And_Economics:_A_Matlab-Based_Introduction_by_Paolo_Brandimarte/CH7/EX7.2/Page_409_CompLatticeBLS.R}
\end{code}

\vspace*{10mm}
\curlable{Exa~7.3}
\begin{code}
\tcaption{Price a pay later option by a binomial lattice}{Price a pay later option by a binomial lattice}
\lstinputlisting{../Numerical_Methods_In_Finance_And_Economics:_A_Matlab-Based_Introduction_by_Paolo_Brandimarte/CH7/EX7.3/Page_411_Luenberger_Investment Science_Ex13_11.R}
\end{code}

\vspace*{10mm}
\curlable{Exa~7.4}
\begin{code}
\tcaption{Pricing an European call by a binomial lattice}{Pricing an European call by a binomial lattice}
\lstinputlisting{../Numerical_Methods_In_Finance_And_Economics:_A_Matlab-Based_Introduction_by_Paolo_Brandimarte/CH7/EX7.4/Page_413_SmartEurLattice.R}
\end{code}

\vspace*{10mm}
\curlable{Exa~7.5}
\begin{code}
\tcaption{Pricing an American put by a binomial lattice}{Pricing an American put by a binomial lattice}
\lstinputlisting{../Numerical_Methods_In_Finance_And_Economics:_A_Matlab-Based_Introduction_by_Paolo_Brandimarte/CH7/EX7.5/Page_416_AmPutLattice.R}
\end{code}

\vspace*{10mm}
\curlable{Exa~7.6}
\begin{code}
\tcaption{Pricing an American spread option by a bidimensional binomial lattice}{Pricing an American spread option by a bidimensional binomial lattice}
\lstinputlisting{../Numerical_Methods_In_Finance_And_Economics:_A_Matlab-Based_Introduction_by_Paolo_Brandimarte/CH7/EX7.6/Page_420_AmSpreadLattice.R}
\end{code}

\vspace*{10mm}
\curlable{Exa~7.7}
\begin{code}
\tcaption{Pricing an European call by a trinomial lattice}{Pricing an European call by a trinomial lattice}
\lstinputlisting{../Numerical_Methods_In_Finance_And_Economics:_A_Matlab-Based_Introduction_by_Paolo_Brandimarte/CH7/EX7.7/Page_424_EuCallTrinomial.R}
\end{code}

\chapter{Option Pricing by Monte Carlo Methods }

\vspace*{10mm}
\curlable{Exa~8.1}
\begin{code}
\tcaption{Generate asset price paths by Monte Carlo simulation}{Generate asset price paths by Monte Carlo simulation}
\lstinputlisting{../Numerical_Methods_In_Finance_And_Economics:_A_Matlab-Based_Introduction_by_Paolo_Brandimarte/CH8/EX8.1/Page_433_AssetPaths.R}
\end{code}

\vspace*{10mm}
\curlable{Exa~8.2}
\begin{code}
\tcaption{Vectorized code to generate asset price paths}{Vectorized code to generate asset price paths}
\lstinputlisting{../Numerical_Methods_In_Finance_And_Economics:_A_Matlab-Based_Introduction_by_Paolo_Brandimarte/CH8/EX8.2/Page_434_AssetPathsV.R}
\end{code}

\vspace*{10mm}
\curlable{Exa~8.3}
\begin{code}
\tcaption{Evaluating the cost of a stop loss hedging strategy}{Evaluating the cost of a stop loss hedging strategy}
\lstinputlisting{../Numerical_Methods_In_Finance_And_Economics:_A_Matlab-Based_Introduction_by_Paolo_Brandimarte/CH8/EX8.3/Page_436_StopLoss.R}
\end{code}

\vspace*{10mm}
\curlable{Exa~8.4}
\begin{code}
\tcaption{Vectorized code for the stop loss hedging strategy}{Vectorized code for the stop loss hedging strategy}
\lstinputlisting{../Numerical_Methods_In_Finance_And_Economics:_A_Matlab-Based_Introduction_by_Paolo_Brandimarte/CH8/EX8.4/Page_437_StopLossV.R}
\end{code}

\vspace*{10mm}
\curlable{Exa~8.5}
\begin{code}
\tcaption{Evaluating the performance of delta hedging}{Evaluating the performance of delta hedging}
\lstinputlisting{../Numerical_Methods_In_Finance_And_Economics:_A_Matlab-Based_Introduction_by_Paolo_Brandimarte/CH8/EX8.5/Page_438_DeltaHedging.R}
\end{code}

\vspace*{10mm}
\curlable{Exa~8.6}
\begin{code}
\tcaption{Implementing and checking path generation for the standard Wiener process by a Brownian bridge}{Implementing and checking path generation for the standard Wiener process by a Brownian bridge}
\lstinputlisting{../Numerical_Methods_In_Finance_And_Economics:_A_Matlab-Based_Introduction_by_Paolo_Brandimarte/CH8/EX8.6/Page_442_WienerBridge.R}
\end{code}

\vspace*{10mm}
\curlable{Exa~8.7}
\begin{code}
\tcaption{Code to price an exchange option analytically}{Code to price an exchange option analytically}
\lstinputlisting{../Numerical_Methods_In_Finance_And_Economics:_A_Matlab-Based_Introduction_by_Paolo_Brandimarte/CH8/EX8.7/Page_444_Exchange.R}
\end{code}

\vspace*{10mm}
\curlable{Exa~8.8}
\begin{code}
\tcaption{Code to price an exchange option by Monte Carlo simulation}{Code to price an exchange option by Monte Carlo simulation}
\lstinputlisting{../Numerical_Methods_In_Finance_And_Economics:_A_Matlab-Based_Introduction_by_Paolo_Brandimarte/CH8/EX8.8/Page_445_ExchangeMC.R}
\end{code}

\vspace*{10mm}
\curlable{Exa~8.9}
\begin{code}
\tcaption{Crude Monte Carlo simulation for a discrete barrier option}{Crude Monte Carlo simulation for a discrete barrier option}
\lstinputlisting{../Numerical_Methods_In_Finance_And_Economics:_A_Matlab-Based_Introduction_by_Paolo_Brandimarte/CH8/EX8.9/Page_446_DOPutMC.R}
\end{code}

\vspace*{10mm}
\curlable{Exa~8.10}
\begin{code}
\tcaption{Conditional Monte Carlo simulation for a discrete barrier option}{Conditional Monte Carlo simulation for a discrete barrier option}
\lstinputlisting{../Numerical_Methods_In_Finance_And_Economics:_A_Matlab-Based_Introduction_by_Paolo_Brandimarte/CH8/EX8.10/Page_449_D0PutMCCond.R}
\end{code}

\vspace*{10mm}
\curlable{Exa~8.11}
\begin{code}
\tcaption{Using conditional Monte Carlo and importance sampling for a discrete barrier option}{Using conditional Monte Carlo and importance sampling for a discrete barrier option}
\lstinputlisting{../Numerical_Methods_In_Finance_And_Economics:_A_Matlab-Based_Introduction_by_Paolo_Brandimarte/CH8/EX8.11/Page_453_DOPutMCCondIS.R}
\end{code}

\vspace*{10mm}
\curlable{Exa~8.12}
\begin{code}
\tcaption{Monte Carlo simulation for an Asian option}{Monte Carlo simulation for an Asian option}
\lstinputlisting{../Numerical_Methods_In_Finance_And_Economics:_A_Matlab-Based_Introduction_by_Paolo_Brandimarte/CH8/EX8.12/Page_455_AsianMC.R}
\end{code}

\vspace*{10mm}
\curlable{Exa~8.13}
\begin{code}
\tcaption{Monte Carlo simulation with control variates for an Asian option}{Monte Carlo simulation with control variates for an Asian option}
\lstinputlisting{../Numerical_Methods_In_Finance_And_Economics:_A_Matlab-Based_Introduction_by_Paolo_Brandimarte/CH8/EX8.13/Page_456_AsianMCCV.R}
\end{code}

\vspace*{10mm}
\curlable{Exa~8.14}
\begin{code}
\tcaption{Using the geometric average Asian option as a control variate}{Using the geometric average Asian option as a control variate}
\lstinputlisting{../Numerical_Methods_In_Finance_And_Economics:_A_Matlab-Based_Introduction_by_Paolo_Brandimarte/CH8/EX8.14/Page_459_AsianMCGeoCV.R}
\end{code}

\vspace*{10mm}
\curlable{Exa~8.15}
\begin{code}
\tcaption{Pricing an Asian option by Halton sequences}{Pricing an Asian option by Halton sequences}
\lstinputlisting{../Numerical_Methods_In_Finance_And_Economics:_A_Matlab-Based_Introduction_by_Paolo_Brandimarte/CH8/EX8.15/Page_460_AsianHalton.R}
\end{code}

\vspace*{10mm}
\curlable{Exa~8.16}
\begin{code}
\tcaption{Simulating geometric Brownian motion by Halton sequences and the Brownian bridge}{Simulating geometric Brownian motion by Halton sequences and the Brownian bridge}
\lstinputlisting{../Numerical_Methods_In_Finance_And_Economics:_A_Matlab-Based_Introduction_by_Paolo_Brandimarte/CH8/EX8.16/Page_464_GBMHaltonBridge.R}
\end{code}

\vspace*{10mm}
\curlable{Exa~8.17}
\begin{code}
\tcaption{Improving the estimate of the option Delta by Common Random Numbers}{Improving the estimate of the option Delta by Common Random Numbers}
\lstinputlisting{../Numerical_Methods_In_Finance_And_Economics:_A_Matlab-Based_Introduction_by_Paolo_Brandimarte/CH8/EX8.17/Page_470_BlsDeltaMCNaive.R}
\end{code}

\vspace*{10mm}
\curlable{Exa~8.18}
\begin{code}
\tcaption{Estimating the option Delta by a pathwise estimator}{Estimating the option Delta by a pathwise estimator}
\lstinputlisting{../Numerical_Methods_In_Finance_And_Economics:_A_Matlab-Based_Introduction_by_Paolo_Brandimarte/CH8/EX8.18/Page_472_BlsDeltaMCPath.R}
\end{code}

\chapter{ Option Pricing by Finite Difference Methods}

\vspace*{10mm}
\curlable{Exa~9.3}
\begin{code}
\tcaption{price a European vanilla put by a straightforward explicit scheme}{price a European vanilla put by a straightforward explicit scheme}
\lstinputlisting{../Numerical_Methods_In_Finance_And_Economics:_A_Matlab-Based_Introduction_by_Paolo_Brandimarte/CH9/EX9.3/Page_479_EuPutExpl.R}
\end{code}

\vspace*{10mm}
\curlable{Exa~9.4}
\begin{code}
\tcaption{price a vanilla European option by a fully implicit method}{price a vanilla European option by a fully implicit method}
\lstinputlisting{../Numerical_Methods_In_Finance_And_Economics:_A_Matlab-Based_Introduction_by_Paolo_Brandimarte/CH9/EX9.4/Page_484_EuPutImpl.R}
\end{code}

\vspace*{10mm}
\curlable{Exa~9.5}
\begin{code}
\tcaption{price a down and out put option by the Crank Nicolson method}{price a down and out put option by the Crank Nicolson method}
\lstinputlisting{../Numerical_Methods_In_Finance_And_Economics:_A_Matlab-Based_Introduction_by_Paolo_Brandimarte/CH9/EX9.5/Page_487_DOPutCK.R}
\end{code}

\chapter{ Dynamic Programming}

\vspace*{10mm}
\curlable{Exa~10.4}
\begin{code}
\tcaption{Simple asset allocation problem under uncertainty Monte Carlo sampling}{Simple asset allocation problem under uncertainty Monte Carlo sampling}
\lstinputlisting{../Numerical_Methods_In_Finance_And_Economics:_A_Matlab-Based_Introduction_by_Paolo_Brandimarte/CH10/EX10.4/Ex10_4.R}
\end{code}

\end{document}
